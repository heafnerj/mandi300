% !TEX TS-program = lualatexmk
% !TEX encoding = UTF-8 Unicode
%
% Minimal Template Version 20200722
%

% MUCH of this code will eventually be in the mandi package so I'm
%  coding this file as though it's a package, not a document.

\documentclass{article}
% unicode-math loads fontspec and xparse
\RequirePackage[math-style=ISO]{unicode-math}
%  Only one of the next two lines need be used.
%\setmathfont{texgyrepagella-math.otf}
%\setmathfont{TeX Gyre Bonum Math}   % Good g for vectors, tensors.
\setmathfont{TeX Gyre DejaVu Math}  % Good g everywhere. Based on Arev!
%\setmathfont{Latin Modern Math}     % Wrong g for tensors.
%\setmathfont{TeX Gyre Pagella Math} % Wrong g for vectors.
%\setmathfont{TeX Gyre Termes Math}  % Wrong g for vectors.
%\setmathfont{TeX Gyre Schola Math}  % Wrong g everywhere.
%  Borrow upright san serif Greek from Fira Math
%  See https://tex.stackexchange.com/a/550697/218142
%  Only one of the next two lines need be used.
%\setmathfont[range=up/{greek,Greek}]{FiraMath-Regular.otf}
%\setmathfont[range=up/{greek,Greek}]{Fira Math}
\setmathfont[range=it/{greek,Greek}]{Latin Modern Math}
\setmathfont[range=bfit/{greek,Greek}]{Latin Modern Math}
\setmathfont[range=up/{greek,Greek}]{Latin Modern Math}
\setmathfont[range=bfup/{greek,Greek}]{Latin Modern Math}
\setmathfont[range=bfsfup/{greek,Greek}]{Latin Modern Math}
%  Borrow \mathscr and \mathbfscr from XITS Math
%  See https://tex.stackexchange.com/a/120073/218142
\setmathfont[range={\mathscr,\mathbfscr}]{XITS Math}
%  Get original and bold \mathcal font
%  See https://tex.stackexchange.com/a/21742/218142
\setmathfont[range={\mathcal,\mathbfcal},StylisticSet=1]{XITS Math}
%  The next two lines are probably not needed.
%\DeclareMathAlphabet{\mathcal}{OMS}{cmsy}{m}{n} % gives original \mathcal font
%\SetMathAlphabet{\mathcal}{bold}{OMS}{cmsy}{b}{n}
\RequirePackage{amsmath}          % AMS goodness; don't load amssymb or amsfonts
%\RequirePackage{braket}           % needed for Dirac notation
%\RequirePackage{derivative}       % needed for nice derivatives
\RequirePackage{esvect}           % needed for nice vector arrow
\RequirePackage{hyperref}
\hypersetup{colorlinks=true,urlcolor=blue}
\RequirePackage{listings}         % needed for program listings
\RequirePackage{mathtools}        % needed for paired delimiters; extends amsmath
\RequirePackage{nicematrix}       % needed for nice matrices
\RequirePackage{tensor}           % needed for index notation

% Settings for code listings.
\definecolor{vbgcolor}{rgb}{1,1,1}           % background for code listings
\definecolor{vshadowcolor}{rgb}{0.5,0.5,0.5} % shadow for code listings
\lstdefinestyle{vpython}{%                   % style for code listings
  language=Python,%                          % select language
  morekeywords={__future__,division,append,  % VPython/GlowScript specific keywords
  arange,arrow,astuple,axis,background,black,blue,cyan,green,magenta,orange,red,%
  white,yellow,border,box,color,comp,cone,convex,cross,curve,cylinder,degrees,%
  diff_angle,dot,ellipsoid,extrusion,faces,font,frame,graphs,headlength,height,%
  headwidth,helix,index,interval,label,length,line,linecolor,mag,mag2,make_trail,%
  material,norm,normal,objects,opacity,points,pos,print,print_function,proj,pyramid,%
  radians,radius,rate,retain,ring,rotate,scene,shaftwidth,shape,sign,size,space,%
  sphere,text,trail_object,trail_type,True,twist,up,vector,visual,width,offset,%
  yoffset,GlowScript,VPython,vpython,trail_color,trail_radius,pps,clear,False,%
  CoffeeScript,graph,gdisplay,canvas,pause,vec,clone,compound,vertex,triangle,quad,%
  attach_trail,attach_arrow,textures,bumpmaps,print_options,get_library,%
  read_local_file},%
  captionpos=b,%                       % position caption
  frame=shadowbox,%                    % shadowbox around listing
  rulesepcolor=\color{vshadowcolor},%  % shadow color
  basicstyle=\footnotesize,%           % basic font for code listings
  commentstyle=\bfseries\color{red},%  % font for comments
  keywordstyle=\bfseries\color{blue},% % font for keywords
  showstringspaces=true,%              % show spaces in strings
  stringstyle=\bfseries\color{green},% % color for strings
  numbers=left,%                       % where to put line numbers
  numberstyle=\tiny,%                  % set to 'none' for no line numbers
  xleftmargin=20pt,%                   % extra left margin
  backgroundcolor=\color{vbgcolor},%   % some people find this annoying
  upquote=true,%                       % how to typeset quotes
  breaklines=true}%                    % break long lines

% These are for demon purposes.
\pagestyle{empty}
\usepackage{geometry}
\geometry{margin=0.5in,top=0.25in}
\usepackage{lipsum}

\newcommand{\reason}[1]{&&\begin{minipage}{5cm}{\small{#1}}\end{minipage}}


%%%%% Redefine some commands to account for index notation.
% We need some good semantic commands to distinguish index notation
% from coordinate free notation. These are experimental.
\NewDocumentCommand{\veccomp}{ s m }{%
  \IfBooleanTF{#1}
  {% We have a *.
    \mathnormal{#2}%
  }%
  {% We don't have a *.
    \mathbfit{#2}%
  }%
}%
\NewDocumentCommand{\tencomp}{ s m }{%
  \IfBooleanTF{#1}
  {% We have a *.
    \mathsfit{#2}%
  }%
  {% We don't have a *.
    \mathbfsfit{#2}%
  }%
}%

%%%%% We need some custom commands for coordinate-free tensors.
%  This is an intelligent slot command for coordinate-free tensor notation.
%\newcommand*{\slot}[1][~]{\,\underline{\smash{\makebox[1.5em]{\ensuremath{#1}}}}\,}
%  \slot       gives just a blank slot 1.0em wide.
%  \slot[...]  fills the slot with a vector but doesn't show the slot.
%  \slot*[...] fills the slot but does show the slot.
%  \slot*      isn't defined and shouldn't be used even though it doesn't 
%               throw an error.
\NewDocumentCommand{\slot}{ s d[] }{%
  % d[] must be used because of the way consecutive optional
  %  arguments are handled. See xparse docs for details.
  \IfBooleanTF{#1}
  {% We have a *.
    \IfValueT{#2}
    {% Insert a vector, and show the slot.
      \underline{\smash{\makebox[1.5em]{\ensuremath{#2}}}}
    }%
  }%
  {% We don't have a *.
    \IfValueTF{#2}
    {% Insert a vector, but don't show the slot.
      \smash{\makebox[1.5em]{\ensuremath{#2}}}
    }%
    {% No vector; just show the slot.
      \underline{\smash{\makebox[1.5em]{\ensuremath{}}}}
    }%
  }%
}%

%  A small slot for coordinate-free contraction notation.
%\newcommand*{\smallslot}{\,\underline{\smash{\makebox[0.04em]{\ensuremath{~}}}}\,}

%  Notation for contraction on pairs of slots.
%\newcommand*{\contraction}[1]{\mathbbmss{C}_{#1}}
%\newcommand*{\contraction}[1]{C_{#1}}
\NewDocumentCommand{\contraction}{ s m }{%
  \IfBooleanTF {#1} % check for *
  {% Use C for contraction.
    \mathsf{C}_{#2}
  }%
  {%
    \symbb{C}_{#2}
  }%  
}%

%  An intelligent exterior derivative operator.
\NewDocumentCommand{\dd}{ s }{%
  \IfBooleanTF{#1}
  {% We have a *.
    \mathop{}\!\symbfsfup{d}
  }%
  {% We don't have a *.
    \mathop{}\!\symsfup{d}
  }%
}%

%  Notation for tensor valence.
%  \binom looks terrible in inline math!
%  See https://tex.stackexchange.com/q/269980/218142
%\newcommand*{\valence}[2]{\ensuremath{\binom{#1}{#2}}}
%  Works, but still sucks.
%\newcommand*{\valence}[2]{\ensuremath{\biggl(\genfrac{}{}{0pt}{}{#1}{#2}\biggr)}}
%  Switch to horizontal.
\newcommand*{\valence}[2]{\ensuremath{(#1,#2)}}
%%%%%%%%%%%%%%%%%%%%%%%%%%%%%%%%%%%%%%%%%%%%%%%%%%%%%%%%%%%%%%%

%%%%% Add float option to code environments and file inputs.
%  There is no \lstrenewenvironment command.
%  CAVEAT: A float can't be split across a page break!
%  Perhaps define a starred version that floats, while
%  the regular version doesn't float, and therefore 
%  breaks nicely across pages.
\lstnewenvironment{newvpythonblock}[2]{%
  \lstset{style=vpython,caption={#1},label={#2}}}{}
%\lstnewenvironment{newglowscriptblock}[2]{%
%  \lstset{style=vpython,caption={#1},label={#2}}}{}
\lstnewenvironment{newglowscriptblock}[3]{%
  \lstset{style=vpython,caption={\href{#3}{#1}},label={#2}}}{}
\newcommand*{\vpythonfile}[3]{%
  \newpage\lstinputlisting[style=vpython,caption={#1},label={#2}]{#3}}
\newcommand*{\glowscriptfile}[3]{%
  \newpage\lstinputlisting[style=vpython,caption={#1},label={#2}]{#3}}
%%%%%%%%%%%%%%%%%%%%%%%%%%%%%%%%%%%%%%%%%%%%%%%%%%%%%%%%%%%%%

\ExplSyntaxOn

%%%%% Row and column vectors
%%%%% ALREADY IN MANDI
\NewDocumentCommand{\Rowvec}{ O{,} m }{%
  % Based on https://tex.stackexchange.com/a/39054/218142.
  \vector_main:nnnn { p } { & } { #1 } { #2 }
}%

\NewDocumentCommand{\Colvec}{ O{,} m }{%
  \vector_main:nnnn { p } { \\ } { #1 } { #2 }
}%

\seq_new:N \l__vector_arg_seq
\cs_new_protected:Npn \vector_main:nnnn #1 #2 #3 #4 {%
  \seq_set_split:Nnn \l__vector_arg_seq { #3 } { #4 }
  \begin{#1NiceMatrix}[r]
    \seq_use:Nnnn \l__vector_arg_seq { #2 } { #2 } { #2 }
  \end{#1NiceMatrix}
}%
%%%%%%%%%%%%%%%%%%%%%%%%%%%%

\ExplSyntaxOff

% Define an intelligent \tns command for tensors?
%%%%% An intelligent \vec command.
%  Why doesn't it work when I put spaces around #3 or #4?
%  Because outside of \ExplSyntaxOn...\ExplSyntaxOff, 
%  the _ character has a different catcode.
%  See https://tex.stackexchange.com/q/554706/218142.
%  This needs to incorporate ISO 80000-2.
\RenewDocumentCommand{\vec}{ s m e{_^} }{%
  \ensuremath{%
    % Note the \, used to make superscript look better.
    \IfBooleanTF {#1}   % check for *
      {\vv{#2}          % * gives an arrow
        % Use \sp{} primitive for superscript.
        % Adjust superscript for the arrow.
        \sp{\IfValueT{#4}{\,#4}\vphantom{\smash[t]{\big|}}}
      }%         
      {\mathbfit{#2}  % no * gives us bold
        % Use \sp{} primitive for superscript.
        % No superscript adjustment needed.
        \sp{\IfValueT{#4}{#4}\vphantom{\smash[t]{\big|}}}
      }% 
    % Use \sb{} primitive for subscript.
    \sb{\IfValueT{#3}{#3}\vphantom{\smash[b]{|}}}
  }%
}%
%%%%%%%%%%%%%%%%%%%%%%%%%%%%%%%%%%%

\begin{document}
%This is the new \verb!mandi! package, which creates an environment specifically for
%introductory physics students. It provides for consistent notation conventions that 
%align with ISO 80000-2 recommendations.
%
%\section{Fonts}
%\subsection{Text Mode Fonts}
%\subsubsection{Default Text Font Parameters}
%To begin with, let's look at the fonts we have at our disposal. Let's look at the 
%default text mode font parameters first.
%
%\begin{center}
%\begin{tabular}{l l}
%\verb!\encodingdefault! & \encodingdefault \\
%\verb!\familydefault!   & \familydefault   \\
%\verb!\seriesdefault!   & \seriesdefault   \\
%\verb!\shapedefault!    & \shapedefault    \\
%\verb!\rmdefault!       & \rmdefault       \\
%\verb!\sfdefault!       & \sfdefault       \\
%\verb!\ttdefault!       & \ttdefault       \\
%\verb!\bfdefault!       & \bfdefault       \\
%\verb!\updefault!       & \updefault       \\
%\verb!\itdefault!       & \itdefault       \\
%\verb!\mddefault!       & \mddefault       \\
%\verb!\sldefault!       & \sldefault       \\
%\verb!\scdefault!       & \scdefault       \\
%\end{tabular}
%\end{center}

%\subsubsection{Text Mode Font Commands}
%Now let's look at the text mode commands for changing fonts and their accompanying
%results. Note that these commands require braced arguments and work only within the 
%scope of the braces.

%\begin{center}
%\begin{tabular}{l}
%Text Mode Commands                                                          \\
%The default normal text is \verb!\textnormal{...}!.                         \\
%\textnormal{abcdefghijklmnopqrstuvwxyzABCDEFGHIJKLMNOPQRSTUVWXYZ0123456789} \\
%To get roman letters, use \verb!\textrm{...}!.                              \\
%\textrm{abcdefghijklmnopqrstuvwxyzABCDEFGHIJKLMNOPQRSTUVWXYZ0123456789}     \\
%To get san serif letters, use \verb!\textsf{...}!.                          \\
%\textsf{abcdefghijklmnopqrstuvwxyzABCDEFGHIJKLMNOPQRSTUVWXYZ0123456789}     \\
%To get typewriter letters, use \verb!\texttt{...}!.                         \\
%\texttt{abcdefghijklmnopqrstuvwxyzABCDEFGHIJKLMNOPQRSTUVWXYZ0123456789}     \\
%To get medium letters, use \verb!\textmd{...}!.                             \\
%\textmd{abcdefghijklmnopqrstuvwxyzABCDEFGHIJKLMNOPQRSTUVWXYZ0123456789}     \\
%To get boldface letters, use \verb!\textbf{...}!.                           \\
%\textbf{abcdefghijklmnopqrstuvwxyzABCDEFGHIJKLMNOPQRSTUVWXYZ0123456789}     \\
%Text up \verb!\textup{...}!                                                 \\
%\textup{abcdefghijklmnopqrstuvwxyzABCDEFGHIJKLMNOPQRSTUVWXYZ0123456789}     \\
%To get italic letters, use \verb!\textit{...}!.                             \\
%\textit{abcdefghijklmnopqrstuvwxyzABCDEFGHIJKLMNOPQRSTUVWXYZ0123456789}     \\
%To get slanted letters, use \verb!\textsl{...}!.                            \\
%\textsl{abcdefghijklmnopqrstuvwxyzABCDEFGHIJKLMNOPQRSTUVWXYZ0123456789}     \\
%To get small capital letters, use \verb!\textsc{...}!.                      \\
%\textsc{abcdefghijklmnopqrstuvwxyzABCDEFGHIJKLMNOPQRSTUVWXYZ0123456789}     \\
%\end{tabular}
%\end{center}
%
%Some of those text mode commands may seem redundant to you, and there are some you 
%will never need to use, so don't worry if they look confusing. Remember that 
%\verb!\textnormal{...}! is the default and all you have to do to get it is start
%typing; no command is necessary unless you have changed something (and that's usually 
%difficult to do).
%
%These text font commands can be used together if you remember to nest the braces 
%correctly. Look at the following table.

%\begin{center}
%\begin{tabular}{l}
%Combined Text Mode Commands                                                      \\
%To get boldface san serif letters, use \verb!\textbf{\textsf{...}}!.             \\
%\textbf{\textsf{abcdefghijklmnopqrstuvwxyzABCDEFGHIJKLMNOPQRSTUVWXYZ0123456789}} \\
%To get boldface italic letters, use \verb!\textbf{\textit{...}}!.                \\
%\textbf{\textit{abcdefghijklmnopqrstuvwxyzABCDEFGHIJKLMNOPQRSTUVWXYZ0123456789}} \\
%To get boldface slanted letters, use \verb!\textbf{\textsl{...}}!.               \\
%\textbf{\textsl{abcdefghijklmnopqrstuvwxyzABCDEFGHIJKLMNOPQRSTUVWXYZ0123456789}} \\
%%To get boldface small capital letters, use \verb!\textbf{\textsc{...}}!.         \\
%%\textbf{\textsc{abcdefghijklmnopqrstuvwxyzABCDEFGHIJKLMNOPQRSTUVWXYZ0123456789}} \\
%\end{tabular}
%\end{center}

%\subsubsection{Emphasizing Words in Text Mode}
%For semantic reasons, it's important to use \verb!\emph{...}! when you want to 
%emphasize a word. Use \verb!\emph{vector}! to emphasize the word \emph{vector} or 
%\verb!\emph{momentum}! to emphasize the word \emph{momentum}. In most documents, 
%\verb!\emph{...}! defaults to italics, but it may default to something else when 
%using a different document class. Let the document class do its work and you will 
%have fewer things to remember. 
%
%\subsubsection{Text Mode Font Switches}
%Next, let's look as the various text mode switches you can use. They are called 
%\emph{switches} rather than \emph{commands} because they take effect immediately 
%and stay in effect until you explicitly turn them off or activate another one, 
%much like a light switch. As so, they do not take braced arguments, or indeed, 
%any arguments at all. Look at the following table.

%\begin{center}
%\begin{tabular}{l}
%Text Mode Switches                                                         \\
%Use \verb!\normalfont! to switch to the default normal font.               \\
%\normalfont abcdefghijklmnopqrstuvwxyzABCDEFGHIJKLMNOPQRSTUVWXYZ0123456789
%  \normalfont                                                              \\
%Use \verb!\rmfamily! to switch to roman letters.                           \\
%\rmfamily abcdefghijklmnopqrstuvwxyzABCDEFGHIJKLMNOPQRSTUVWXYZ0123456789
%  \normalfont                                                              \\
%Use \verb!\sffamily! to switch to sans serif latters.                      \\
%\sffamily abcdefghijklmnopqrstuvwxyzABCDEFGHIJKLMNOPQRSTUVWXYZ0123456789
%  \normalfont                                                              \\
%Use \verb!\ttfamily! to switch to typewriter letters.                      \\
%\ttfamily abcdefghijklmnopqrstuvwxyzABCDEFGHIJKLMNOPQRSTUVWXYZ0123456789
%  \normalfont                                                              \\
%Use \verb!\mdseries! to switch to medium letters.                          \\
%\mdseries abcdefghijklmnopqrstuvwxyzABCDEFGHIJKLMNOPQRSTUVWXYZ0123456789
%  \normalfont                                                              \\
%Use \verb!\bfseries! to switch to boldface letters.                        \\
%\bfseries abcdefghijklmnopqrstuvwxyzABCDEFGHIJKLMNOPQRSTUVWXYZ0123456789
%  \normalfont                                                              \\
%Use \verb!\upshape! to switch to upright letters.                          \\
%\upshape abcdefghijklmnopqrstuvwxyzABCDEFGHIJKLMNOPQRSTUVWXYZ0123456789
%  \normalfont                                                              \\
%Use \verb!\itshape! to switch to italic letters.                           \\
%\itshape abcdefghijklmnopqrstuvwxyzABCDEFGHIJKLMNOPQRSTUVWXYZ0123456789
%  \normalfont                                                              \\
%Use \verb!\slshape! to switch to slanted letters.                          \\
%\slshape abcdefghijklmnopqrstuvwxyzABCDEFGHIJKLMNOPQRSTUVWXYZ0123456789
%  \normalfont                                                              \\
%Use \verb!\scshape! to switch to small capital letters.                    \\
%\scshape abcdefghijklmnopqrstuvwxyzABCDEFGHIJKLMNOPQRSTUVWXYZ0123456789
%  \normalfont                                                              \\
%\end{tabular}
%\end{center}

%To illustrate how these switches work, the following paragraph is in the default 
%text font (it's just filler text). 
%
%\noindent Lorem ipsum dolor sit amet, consectetur adipiscing elit. Duis egestas urna 
%et dolor posuere, accumsan faucibus justo viverra. Pellentesque libero neque, maximus 
%vitae placerat eu, luctus sit amet sapien. Class aptent taciti sociosqu ad litora 
%torquent per conubia nostra, per inceptos himenaeos. Aenean vel nisl massa. Sed id 
%velit tellus. Vivamus eu elit a erat aliquam auctor nec sed mi. Orci varius natoque 
%penatibus et magnis dis parturient montes, nascetur ridiculus mus. Cras ut consequat 
%purus. Fusce imperdiet scelerisque sagittis. Etiam maximus sagittis sapien. Proin mi 
%metus, cursus a ex eget, maximus laoreet sapien. In hac habitasse platea dictumst. 
%Morbi eget est dui. Cras posuere nisl quis leo facilisis, vel viverra augue sagittis. 
%Nullam posuere, ex id efficitur mollis, velit nisi tincidunt augue, vel tempor orci 
%nisi a justo.
%
%The following paragraph is the same filler text, but now preceded by the 
%\verb!\sffamily! switch.
%
%\sffamily
%\noindent Lorem ipsum dolor sit amet, consectetur adipiscing elit. Duis egestas urna 
%et dolor posuere, accumsan faucibus justo viverra. Pellentesque libero neque, maximus 
%vitae placerat eu, luctus sit amet sapien. Class aptent taciti sociosqu ad litora 
%torquent per conubia nostra, per inceptos himenaeos. Aenean vel nisl massa. Sed id 
%velit tellus. Vivamus eu elit a erat aliquam auctor nec sed mi. Orci varius natoque 
%penatibus et magnis dis parturient montes, nascetur ridiculus mus. Cras ut consequat 
%purus. Fusce imperdiet scelerisque sagittis. Etiam maximus sagittis sapien. Proin mi 
%metus, cursus a ex eget, maximus laoreet sapien. In hac habitasse platea dictumst. 
%Morbi eget est dui. Cras posuere nisl quis leo facilisis, vel viverra augue sagittis. 
%Nullam posuere, ex id efficitur mollis, velit nisi tincidunt augue, vel tempor orci 
%nisi a justo. 
%
%Note that this paragraph is still typeset in sans serif letters because the 
%\verb!\sffamily! switch is still activated. We can deactivate it by activating 
%another switch, like \verb!\normalfont! \normalfont to get back to normal default 
%text, or like \verb!\bfseries! \bfseries to get bold letters, or even like 
%\verb!\itshape! \itshape to get italic letters. Note that we got both bold and italic 
%letters in that last phrase, and in this sentence, because the \verb!\bfseries! switch 
%was still activated when we activated the \verb!\itshape! switch. Activated switches 
%can accumulate to give us new combinations of letter styles and shapes. We can return 
%to the normal default font by activating the \verb!\normalfont! switch \normalfont 
%just like this. Now everything is back to normal.

%\subsection{Math Mode Fonts}
%Now we need to discuss math mode fonts, which are considerably more complicated than 
%text mode fonts. There are math mode font \emph{commands}, but there are no math mode 
%font \emph{switches}. Math mode fonts sometimes include lowercase and/or uppercase 
%Greek letters, which are frequently used in mathematical notation. Math mode fonts 
%go into effect whenever you enter inline math mode with \verb!\(...\)! or display 
%math mode with \verb!\[...\]!. Many \LaTeX{} tutorials use \verb!$...$! for inline 
%math mode and \verb!$$...$$! for display math mode. For various technical reasons, 
%these are not recommended so please do not use them. Please consistently use 
%\verb!\(...\)! for inline math mode and \verb!\[...\]! for display math mode.
%
%\subsubsection{Math Mode Font Commands}
%We have a problem in math mode, because we use the \verb!isomath!, \verb!amsmath!, 
%\verb!amsfonts!, \verb!amsfonts!, \verb!bbm!, \verb!bm!, and \verb!upgreek! packages 
%to create new font combinations that vary from those without loading that packages. 
%The math mode font commands available to you in this environment are given in the 
%following table.
%
%\begin{center}
%\begin{tabular}{l}
%Use \verb!\(...\)! or \verb!\[...\]! to get the default math mode font. \\
%Also use this for vector index notation.                                \\
%\( 
%  abcdefghijklmnopqrstuvwxyzABCDEFGHIJKLMNOPQRSTUVWXYZ0123456789 
%\) \\
%\( 
%  \alpha\beta\gamma\delta\epsilon\varepsilon\zeta\eta\theta\vartheta\iota\kappa
%  \lambda\mu\nu\xi o\pi\varpi\rho\varrho\sigma\varsigma\tau\upsilon\phi\varphi
%  \chi\psi\omega\Delta\Gamma\Theta\Lambda\Xi\Pi\Sigma\Upsilon\Phi\Psi\Omega 
%\) \\
%Use \verb!\mathrm{...}! to get roman letters. This style doesn't have lowercase 
%Greek letters.                                                             \\
%\( 
%  \mathrm{abcdefghijklmnopqrstuvwxyzABCDEFGHIJKLMNOPQRSTUVWXYZ0123456789} 
%\) \\
%\( 
%  \symup{\Delta\Gamma\Theta\Lambda\Xi\Pi\Sigma\Upsilon\Phi\Psi\Omega} 
%\) \\
%Use \verb!\mathbf{...}! to get boldface letters. This style doesn't have lowercase 
%Greek letters.                                                             \\
%Note that this style gives upright letters and numbers.                    \\
%\( 
%  \mathbf{abcdefghijklmnopqrstuvwxyzABCDEFGHIJKLMNOPQRSTUVWXYZ0123456789} 
%\) \\
%\( 
%  \symbfup{\Delta\Gamma\Theta\Lambda\Xi\Pi\Sigma\Upsilon\Phi\Psi\Omega} 
%\) \\
%Use \verb!\mathsf{...}! to get sans serif letters. This style doesn't have lowercase 
%Greek letters.                                                          \\
%Also use this to typeset physical dimensions.                           \\
%\( 
%  \mathsf{abcdefghijklmnopqrstuvwxyzABCDEFGHIJKLMNOPQRSTUVWXYZ0123456789} 
%\) \\
%\( 
%  \mathsf{\Delta\Gamma\Theta\Lambda\Xi\Pi\Sigma\Upsilon\Phi\Psi\Omega} 
%  \symup{\Delta\Gamma\Theta\Lambda\Xi\Pi\Sigma\Upsilon\Phi\Psi\Omega}
%\) \\
%Use \verb!\mathbfit{...}! and \verb!\matrixsym{...}! for vectors and matrices. \\
%Also aliased as \verb!\mathbfit{...}! and \verb!\boldsymbol{...}!. \\
%\( 
%  \mathbfit{abcdefghijklmnopqrstuvwxyzABCDEFGHIJKLMNOPQRSTUVWXYZ0123456789} 
%\) \\
%\( 
%  \mathbfit{\alpha\beta\gamma\delta\epsilon\varepsilon\zeta\eta\theta\vartheta
%  \iota\kappa\lambda\mu\nu\xi o\pi\varpi\rho\varrho\sigma\varsigma\tau\upsilon
%  \phi\varphi\chi\psi\omega\Delta\Gamma\Theta\Lambda\Xi\Pi\Sigma\Upsilon\Phi
%  \Psi\Omega} 
%\) \\
%Use \verb!\mathsfit{...}! for tensor index notation.                    \\
%\( 
%  \mathsfit{abcdefghijklmnopqrstuvwxyzABCDEFGHIJKLMNOPQRSTUVWXYZ0123456789} 
%\) \\
%\( 
%  \mathsfit{\alpha\beta\gamma\delta\epsilon\varepsilon\zeta\eta\theta\vartheta
%  \iota\kappa\lambda\mu\nu\xi o\pi\varpi\rho\varrho\sigma\varsigma\tau\upsilon
%  \phi\varphi\chi\psi\omega\Delta\Gamma\Theta\Lambda\Xi\Pi\Sigma\Upsilon\Phi
%  \Psi\Omega} 
%\) \\
%Use \verb!\mathbfsfit{...}! for tensors.                                  \\
%Also aliased as \verb!\mathbfsfit{...}!.                                 \\
%\( \mathbfsfit{abcdefghijklmnopqrstuvwxyzABCDEFGHIJKLMNOPQRSTUVWXYZ0123456789} \) \\
%\( \mathbfsfit{\alpha\beta\gamma\delta\epsilon\varepsilon\zeta\eta\theta\vartheta
%   \iota\kappa\lambda\mu\nu\xi o\pi\varpi\rho\varrho\sigma\varsigma\tau\upsilon
%   \phi\varphi\chi\psi\omega\Delta\Gamma\Theta\Lambda\Xi\Pi\Sigma\Upsilon\Phi
%   \Psi\Omega} \)                                                       \\
%Use \verb!\up<...>! to name subatomic particles.                        \\
%Also, \verb!\upOmega! represents the ohm.                               \\
%\( 
%  \upalpha\upbeta\upgamma\updelta\upepsilon\upzeta\upeta\uptheta
%  \upvartheta\upiota\upkappa\uplambda\upmu\upnu\upxi\uppi\upvarpi\uprho
%  \upsigma\uptau\upupsilon\upphi\upvarphi\upchi\uppsi\upomega\upDelta
%  \upGamma\upTheta\upLambda\upXi\upPi\upSigma\upUpsilon\upPhi\upPsi\upOmega 
%\) \\
%Use \verb!\symbb{...}! to get sets of numbers and the contraction tensor symbol.\\
%\( 
%  \symbb{abcdefghijklmnopqrstuvwxyzABCDEFGHIJKLMNOPQRSTUVWXYZ} 
%\) \\
%Use \verb!\mathcal{...}!, which only applies to uppercase letters, to get these
%distinctive letters.                                                    \\
%Thorne and Blandford use this font to name points on a manifold.        \\
%It may also be useful for labeling coordinate systems.                  \\
%\(
%  \mathcal{ABCDEFGHIJKLMNOPQRSTUVWXYZ}
%\) \\                                  
%\(
%  \mathbfcal{ABCDEFGHIJKLMNOPQRSTUVWXYZ}
%\) \\                                  
%Here's a math script font if you need it. \\
%\(
%  \mathscr{abcdefghijklmnopqrstuvwxyzABCDEFGHIJKLMNOPQRSTUVWXYZ}
%\) \\                                  
%\(
%  \mathbfscr{abcdefghijklmnopqrstuvwxyzABCDEFGHIJKLMNOPQRSTUVWXYZ}
%\)                                   
%\end{tabular}
%\end{center}
%
%The \verb!\mathbfit{...}! example above is a bit misleading for numbers. The only
%number that is ever used as the name of a vector quantity is zero, and when a 
%\emph{single digit} is used as such, it is typeset upright. That means 
%\verb!\mathbf{0}! gives \( \mathbf{0} \) as expected.

%\subsubsection{Unicode Font Commands}
%Let's test the \texttt{unicode-math} font commands.
%
%\begin{center}
%\begin{tabular}{l}
%  Use \verb!\(...\)! or \verb!\[...\]! or \verb!\symnormal{...}! to get the 
%  default italic math mode font. \\
%  Use this for vector index notation. Does not apply to numerals; 
%  they are always upright. \\
%  \( abcdefghijklmnopqrstuvwxyzABCDEFGHIJKLMNOPQRSTUVWXYZ0123456789 \)    \\
%  \( 
%  \alpha\beta\gamma\delta\epsilon\varepsilon\zeta\eta\theta\vartheta\iota\kappa
%  \lambda\mu\nu\xi o\pi\varpi\rho\varrho\sigma\varsigma\tau\upsilon\phi\varphi
%  \chi\psi\omega\Delta\Gamma\Theta\Lambda\Xi\Pi\Sigma\Upsilon\Phi\Psi\Omega 
%  \) \\
%  Use \verb!\symbf{...}! for boldface italic. Does not apply to numerals. Use 
%  this for vectors and matrices. \\
%  \( \symbf{abcdefghijklmnopqrstuvwxyzABCDEFGHIJKLMNOPQRSTUVWXYZ} \)              \\
%  \( \symbf{ 
%  \alpha\beta\gamma\delta\epsilon\varepsilon\zeta\eta\theta\vartheta\iota\kappa
%  \lambda\mu\nu\xi o\pi\varpi\rho\varrho\sigma\varsigma\tau\upsilon\phi\varphi
%  \chi\psi\omega\Delta\Gamma\Theta\Lambda\Xi\Pi\Sigma\Upsilon\Phi\Psi\Omega }
%  \) \\
%  Use \verb!\symup{...}! for upright serif. \\
%  \( \symup{abcdefghijklmnopqrstuvwxyzABCDEFGHIJKLMNOPQRSTUVWXYZ0123456789} \)      \\
%  \( \symup{ 
%  \alpha\beta\gamma\delta\epsilon\varepsilon\zeta\eta\theta\vartheta\iota\kappa
%  \lambda\mu\nu\xi o\pi\varpi\rho\varrho\sigma\varsigma\tau\upsilon\phi\varphi
%  \chi\psi\omega\Delta\Gamma\Theta\Lambda\Xi\Pi\Sigma\Upsilon\Phi\Psi\Omega }
%  \) \\
%  Use \verb!\symbfup{...}! for boldface upright serif. \\
%  \( \symbfup{abcdefghijklmnopqrstuvwxyzABCDEFGHIJKLMNOPQRSTUVWXYZ0123456789} \)    \\
%  \( \symbfup{ 
%  \alpha\beta\gamma\delta\epsilon\varepsilon\zeta\eta\theta\vartheta\iota\kappa
%  \lambda\mu\nu\xi o\pi\varpi\rho\varrho\sigma\varsigma\tau\upsilon\phi\varphi
%  \chi\psi\omega\Delta\Gamma\Theta\Lambda\Xi\Pi\Sigma\Upsilon\Phi\Psi\Omega }
%  \) \\
%  Use \verb!\symsfup{...}! for sans serif upright. Does not apply to Greek letters. \\
%  Use this to typeset physical dimensions. \\
%  \( \symsfup{abcdefghijklmnopqrstuvwxyzABCDEFGHIJKLMNOPQRSTUVWXYZ0123456789} \)    \\
%  Use \verb!\symbfsfup{...}! for boldface sans serif upright. Applies to Greek
%  letters. \\
%  \( \symbfsfup{abcdefghijklmnopqrstuvwxyzABCDEFGHIJKLMNOPQRSTUVWXYZ0123456789} \)  \\
%  \( \symbfsfup{ 
%  \alpha\beta\gamma\delta\epsilon\varepsilon\zeta\eta\theta\vartheta\iota\kappa
%  \lambda\mu\nu\xi o\pi\varpi\rho\varrho\sigma\varsigma\tau\upsilon\phi\varphi
%  \chi\psi\omega\Delta\Gamma\Theta\Lambda\Xi\Pi\Sigma\Upsilon\Phi\Psi\Omega }
%  \) \\
%  Use \verb!\symsfit{...}! for sans serif italic. \\
%  Use this for tensor index notation. Does not apply to Greek letters.  \\
%  Does not apply to numerals; they are always upright. \\
%  \( \symsfit{abcdefghijklmnopqrstuvwxyzABCDEFGHIJKLMNOPQRSTUVWXYZ} \)              \\
%  Use \verb!\symbfsfit{...}! for boldface sans serif italic. Use this for tensors.  \\
%  \( \symbfsfit{abcdefghijklmnopqrstuvwxyzABCDEFGHIJKLMNOPQRSTUVWXYZ} \)            \\
%%  \verb!\symsf{...}! seems identical to \verb!\symsfit{...}!. \\
%%  \( \symsf{abcdefghijklmnopqrstuvwxyzABCDEFGHIJKLMNOPQRSTUVWXYZ} \)                \\
%%  \verb!\symbfsf{...}! seems identical to \verb!\symbfsfit!. \\
%%  \( \symbfsf{abcdefghijklmnopqrstuvwxyzABCDEFGHIJKLMNOPQRSTUVWXYZ} \)              \\
%  Use \verb!\symcal{...}! for, say, naming points on a manifold. \\
%  \( \symcal{ABCDEFGHIJKLMNOPQRSTUVWXYZ} \)                                         \\
%  Use \verb!\symbfcal{...}! this for naming points on a manifold too. \\
%  \( \symbfcal{ABCDEFGHIJKLMNOPQRSTUVWXYZ} \)                                       \\
%  Use \verb!\symscr{...}! to get script letters. Does not apply to Greek letters.   \\
%  \( \symscr{abcdefghijklmnopqrstuvwxyzABCDEFGHIJKLMNOPQRSTUVWXYZ} \)               \\
%  Use \verb!\symbfscr{...}! to get boldface script letters. Does not apply to 
%  Greek letters. Does not apply to numerals. \\
%  \( \symbfscr{abcdefghijklmnopqrstuvwxyzABCDEFGHIJKLMNOPQRSTUVWXYZ} \)             \\
%  Use \verb!\symtt{...}! to get typewriter letters. Does not apply to Greek letters.\\
%  \( \symtt{abcdefghijklmnopqrstuvwxyzABCDEFGHIJKLMNOPQRSTUVWXYZ0123456789} \)      \\
%  Use \verb!\symfrak{...}! to get Fraktur letters. Does not apply to Greek letters.
%  Does not apply to numerals. \\
%  \( \symfrak{abcdefghijklmnopqrstuvwxyzABCDEFGHIJKLMNOPQRSTUVWXYZ} \)              \\
%  Use \verb!\symbffrak{...}! to get boldface Fraktur letters. Does not apply to     
%  Greek letters. Does not apply to numerals. \\
%  \( \symbffrak{abcdefghijklmnopqrstuvwxyzABCDEFGHIJKLMNOPQRSTUVWXYZ} \)            \\
%  Use \verb!\symbb{...}! to get blackboard letters. Does not apply to Greek 
%  letters. \\
%  \( \symbb{abcdefghijklmnopqrstuvwxyzABCDEFGHIJKLMNOPQRSTUVWXYZ0123456789} \)      \\
%  Use \verb!\symbbit{...}! to get blackboard italic letters. Only applies to five
%  letters. \\
%  \( \symbbit{deijD} \)                                                             \\
%\end{tabular}
%\end{center}

%There's a handful of other mathematical symbols you will need, and here they are. 
%Don't worry if you've never seen any or all of these before. The third row of 
%symbols is unique to \verb!mandi!.
%
%\begin{center}
%\begin{tabular}{l}
%\(
%  \nabla\,\mdlgwhtsquare\,\partial\,\otimes\,\times\,\cdot\,\bullet
%\) \\
%\(
%  \symbf{\nabla}\,\symbf{\mdlgwhtsquare}\,\symbf{\partial}\,\symbf{\otimes}\,
%  \symbf{\times}\,\vysmblkcircle\,\mdlgblkcircle
%\) \\
%\(
%  \nabla\indices{_{\veccomp{u}}}\,\nabla\indices{_{\tencomp{u}}}
%\) \\
%\(
%  \slot\,\slot*[\veccomp{a}]\,\contraction{(1,2)}\,\contraction{(1,3)(2,4)}\,
%  \contraction*{(1,2)}\,\contraction*{(1,3)(2,4)}
%\) \\
%\end{tabular}
%\end{center}

There are four ways to write a vector quantity (in this case, a first rank 
contravariant, or \valence{1}{0}, tensor).

\begin{align*}
  \veccomp{a} &= \veccomp*{a}\indices{^i}\veccomp{e}\indices{_i} 
                 &&\text{ with a basis, using vector symbols }    \\
  \tencomp{a} &= \tencomp*{a}\indices{^i}\tencomp{e}\indices{_i} 
                 &&\text{ with a basis, using tensor symbols }    \\
  \veccomp{a} &= \veccomp{a}(\slot) 
                 &&\text{ coordinate-free, using vector symbols } \\
  \tencomp{a} &= \tencomp{a}(\slot) 
                 &&\text{ coordinate-free, using tensor symbols } \\
\end{align*}

There are five ways to write a tensor quantity (in this case, a second rank 
contravariant, or \valence{2}{0}, tensor). 

\begin{align*}
  \tencomp{T} &= \tencomp{T}(\slot,\slot) &&\text{ coordinate-free, with slots } \\
  \tencomp{T} &= \tencomp*{T}\indices{^i^j}\tencomp{e}\indices{_i}\otimes
                 \tencomp{e}\indices{_j} &&\text{ with a basis } \\
  \tencomp{T} &= \tencomp*{T}\indices{^i^j} &&\text{ omitting the basis } \\
  \tencomp{T} &= \tencomp{Tensor}(\slot,\slot) 
                 &&\text{ coordinate free, named, with slots }
\end{align*}

It probably seems strange to use a word as the name of a tensor 
(e.g.\ \(\symbfsfit{Riemann}\), \(\symbfsfit{Faraday}\), \(\symbfsfit{Maxwell}\), 
\(\symbfsfit{Einstein}\)), but MTW introduced that convention and it most 
definitely has merit.

Showing tensor contraction is challenging because there is no standardized 
coordinate-free notation for it. Therefore, we introduce such a notation here. 
It may surprise you see that this is all equivalent to the dot product of the two 
vectors, but you will see how and why that is very soon. Ultimately, these are all 
just different ways of writing the same real number.

\begin{gather*}
  \contraction{(1,2)}\veccomp{a}(\slot)\otimes\veccomp{b}(\slot) \\
  \contraction{(1,2)}\veccomp{b}(\slot)\otimes\veccomp{a}(\slot) \\
  \veccomp{a}(\slot[\veccomp{b}])  \\
  \veccomp{b}(\slot[\veccomp{a}])  \\
  \veccomp{a}(\slot*[\veccomp{b}]) \\
  \veccomp{b}(\slot*[\veccomp{a}]) \\
  \tencomp{dot}(\slot[\veccomp{a}],\slot[\veccomp{b}])      \\ 
  \tencomp{dot}(\slot[\veccomp{b}],\slot[\veccomp{a}])      \\
  \tencomp{dot}(\slot*[\veccomp{a}],\slot*[\veccomp{b}])    \\
  \tencomp{dot}(\slot*[\veccomp{b}],\slot*[\veccomp{a}])    \\
  \tencomp{metric}(\slot[\veccomp{a}],\slot[\veccomp{b}])   \\ 
  \tencomp{metric}(\slot[\veccomp{b}],\slot[\veccomp{a}])   \\
  \tencomp{metric}(\slot*[\veccomp{a}],\slot*[\veccomp{b}]) \\ 
  \tencomp{metric}(\slot*[\veccomp{b}],\slot*[\veccomp{a}]) \\
  \tencomp{g}(\slot[\veccomp{a}],\slot[\veccomp{b}])   \\
  \tencomp{g}(\slot[\veccomp{b}],\slot[\veccomp{a}])   \\
  \tencomp{g}(\slot*[\veccomp{a}],\slot*[\veccomp{b}]) \\
  \tencomp{g}(\slot*[\veccomp{b}],\slot*[\veccomp{a}]) \\
  \veccomp{a}\cdot\veccomp{b} \\
  \veccomp{b}\cdot\veccomp{a} \\
  \symbb{R} \\
\end{gather*}

As you can see there is a lot of flexibility in how tensor slots are typeset. This 
design is intentional and I hope to see this notation propagated into textbooks at 
every level.

\begin{gather*}
  \tencomp{T}(\slot,\slot) \\
  \tencomp{T}(\slot[\veccomp{a}],\slot)  \\
  \tencomp{T}(\slot*[\veccomp{a}],\slot) \\
  \tencomp{T}(\slot,\slot[\veccomp{b}])  \\
  \tencomp{T}(\slot,\slot*[\veccomp{b}]) \\
  \tencomp{T}(\slot[\veccomp{a}],\slot[\veccomp{b}]) \\
  \tencomp{T}(\slot*[\veccomp{a}],\slot*[\veccomp{b}]) \\
\end{gather*}

An important issue is how a filled slot is visualized. You can choose to either 
show it explicitly or not show it at all. I tend to prefer to explicitly show a 
filled slot, but your preferences may vary.

\begin{gather*}
  \symbfsfup{T}(\slot,\symbfsfup{S}(\symbfsfup{B},\slot),\slot) \\
  \tencomp{T}  (\slot,\slot[\tencomp{S}(\tencomp{B}],\slot),\slot) \\
\end{gather*}

Notation for filling slots with quantities having their own slots can get 
complicated, and perhaps this is something to be considered.

\section{Row and Column Vectors}
\subsection{Column Vectors}
\[
  \Colvec{0,1,2,3}\Colvec{\tencomp*{a}_1,\tencomp*{a}_2,\tencomp*{a}_2}
\]

\subsection{Row Vectors}
\[
  \Rowvec{0,1,2,3}\Rowvec{\tencomp*{a}_1,\tencomp*{a}_2,\tencomp*{a}_2}
\]

\section{Symbolic Vectors}
\[
  \vec{E}_{\textnormal{ball}}^{\textnormal{one}}
\]
\[
  \vec*{E}_{\textnormal{ball}}^{\textnormal{one}}
\]

%\section{Program Listings}
%\texttt{GlowScript} and \texttt{VPython} program listings now float so they won't 
%split across page breaks as often now. However, there is still the problem that if 
%a listing is too long for the floating window it will be truncated. I'll provide 
%two environments to address this problem.


%\begin{newglowscriptblock}{Example Program}{pref1}
%GlowScript 3.0 VPython
%
%# a Young's modulus problem
%
%Lo = 3   # wire's original length in m
%d = 3e-3 # wire's diameter in m
%g = 9.8  # surface grav. field strength in N/kg
%m = 10   # ball's mass in kg
%Y = 2e11 # steel's Young's modulus in N/m^2
%# Find DeltaL, the amount the wire stretches.
%
%area = (pi*d**2)/4
%Force = m*g
%DeltaL = (Force*Lo)/(area*Y)
%
%print ("The wire stretches by ",DeltaL," m")
%\end{newglowscriptblock}


%\begin{newglowscriptblock}{Problem 37 Code}{pref2}{https://tinyurl.com/y5o2ckzn}
%GlowScript 3.0 VPython
%
%# a Young's modulus problem
%
%Lo = 3   # wire's original length in m
%d = 3e-3 # wire's diameter in m
%g = 9.8  # surface grav. field strength in N/kg
%m = 10   # ball's mass in kg
%Y = 2e11 # steel's Young's modulus in N/m^2
%# Find DeltaL, the amount the wire stretches.
%
%area = (pi*d**2)/4
%Force = m*g
%DeltaL = (Force*Lo)/(area*Y)
%
%print ("The wire stretches by ",DeltaL," m")
%\end{newglowscriptblock}
%
%
%
%\begin{gather*}
%  ∇∙𝑬=\frac{𝜌}{𝜀_o} \\
%    ∇∙𝑩=0 \\
%    ∇×𝑬=−\frac{∂𝑩}{∂𝑡} \\
%    ∇×𝑩=𝜇_o \biggl(I +  𝜀_o \frac{∂𝑬}{∂𝑡}\biggr)
%\end{gather*}
%
%\[
%  \dd*\omega
%\]

\begin{align}
  \Delta\vec{r} &= \vec{r}_{\textnormal{final}} - \vec{r}_{\textnormal{initial}} \reason{definition} \\
  \Delta\vec*{r} &= \vec*{r}_{\textnormal{final}} - \vec*{r}_{\textnormal{initial}} 
    \reason{\lipsum[2][1-3]}\\
  \Delta\vec{r} &= \vec{r}_{\textnormal{final}} - \vec{r}_{\textnormal{initial}} \reason{definition} \\
  \Delta\vec{r} &= \vec{r}_{\textnormal{final}} - \vec{r}_{\textnormal{initial}} \reason{definition}
\end{align}

\end{document}
