% !TEX TS-program = lualatexmk
% !TEX encoding = UTF-8 Unicode
%
% Minimal Template Version 20201130
%

\documentclass{article}
\usepackage{mandi300}         % main mandi package
\mandisetup{}                 % mandi package options
\usepackage{mandiexp300}      % expressions for M&I; optional
\hypersetup{colorlinks=false} % borders around links
%\hypersetup{colorlinks=true} % colored links; no borders

% These are for demon purposes.
\usepackage{geometry}         % needed for changing margins
\usepackage{lipsum}           % needed for dummy text

% See https://tex.stackexchange.com/q/156383/218142
\newcommand*{\pkg}[1]{\textsf{#1}}                    % typeset package names
\newcommand*{\GlowScript}{\texttt{GlowScript}}        % typeset GlowScript
\newcommand*{\GlowScriptorg}{\texttt{GlowScript.org}} % typeset GlowScript.org
\newcommand*{\VPython}{\texttt{VPython}}              % typeset VPython
\newcommand*{\VPythonorg}{\texttt{VPython.org}}       % typeset VPython.org
\newcommand*{\lualatex}{Lua\LaTeX}                    % typeset LuaLaTeX

% We need a new command for in-line listings to prevent overfull boxes.
% Anything in |...| will be in small plain text.
% Previously used !...! but that conflicts with colors.
\lstMakeShortInline[basicstyle=\normalfont\ttfamily\small]|

\begin{document}
\tableofcontents
\lstlistoflistings
\newpage

% We need narrower margins to demonstrate certain fonts and features.
% Remember \newgeometry forces a page break.
\newgeometry{left=0.5in,right=0.5in}
This is the new 
\href{https://www.ctan.org/pkg/mandi}{\pkg{mandi}} 
package (\mandiversion), which creates a habitat specifically for introductory physics students. 
It provides consistent notation conventions that align with ISO 80000-2 recommendations. This document
is a demonstration for the new package and should not be taken as official documentation. Features
are not necessarily final and are still subject to change.

\section{Fonts}
\subsection{Text Mode Fonts}
\subsubsection{Default Text Font Parameters}
To begin with, let's look at the fonts we have at our disposal. Let's look at the 
default text mode font parameters first.

\begin{center}
  \begin{tabular}{l l}
    |\encodingdefault| & \encodingdefault \\
    |\familydefault|   & \familydefault   \\
    |\seriesdefault|   & \seriesdefault   \\
    |\shapedefault|    & \shapedefault    \\
    |\rmdefault|       & \rmdefault       \\
    |\sfdefault|       & \sfdefault       \\
    |\ttdefault|       & \ttdefault       \\
    |\bfdefault|       & \bfdefault       \\
    |\updefault|       & \updefault       \\
    |\itdefault|       & \itdefault       \\
    |\mddefault|       & \mddefault       \\
    |\sldefault|       & \sldefault       \\
    |\scdefault|       & \scdefault       \\
  \end{tabular}
\end{center}

\subsubsection{Text Mode Font Commands}
Now let's look at the text mode commands for changing fonts and their accompanying
results. Note that these commands require braced arguments and work only within the 
scope of the braces.
\begin{center}
  \begin{tabular}{l}
    Text Mode Commands                                                          \\
    The default normal text is |\textnormal{...}|.                         \\
    \textnormal{abcdefghijklmnopqrstuvwxyzABCDEFGHIJKLMNOPQRSTUVWXYZ0123456789} \\
    To get roman letters, use |\textrm{...}|.                              \\
    \textrm{abcdefghijklmnopqrstuvwxyzABCDEFGHIJKLMNOPQRSTUVWXYZ0123456789}     \\
    To get san serif letters, use |\textsf{...}|.                          \\
    \textsf{abcdefghijklmnopqrstuvwxyzABCDEFGHIJKLMNOPQRSTUVWXYZ0123456789}     \\
    To get typewriter letters, use |\texttt{...}|.                         \\
    \texttt{abcdefghijklmnopqrstuvwxyzABCDEFGHIJKLMNOPQRSTUVWXYZ0123456789}     \\
    To get medium letters, use |\textmd{...}|.                             \\
    \textmd{abcdefghijklmnopqrstuvwxyzABCDEFGHIJKLMNOPQRSTUVWXYZ0123456789}     \\
    To get boldface letters, use |\textbf{...}|.                           \\
    \textbf{abcdefghijklmnopqrstuvwxyzABCDEFGHIJKLMNOPQRSTUVWXYZ0123456789}     \\
    Text up |\textup{...}|                                                 \\
    \textup{abcdefghijklmnopqrstuvwxyzABCDEFGHIJKLMNOPQRSTUVWXYZ0123456789}     \\
    To get italic letters, use |\textit{...}|.                             \\
    \textit{abcdefghijklmnopqrstuvwxyzABCDEFGHIJKLMNOPQRSTUVWXYZ0123456789}     \\
    To get slanted letters, use |\textsl{...}|.                            \\
    \textsl{abcdefghijklmnopqrstuvwxyzABCDEFGHIJKLMNOPQRSTUVWXYZ0123456789}     \\
    To get small capital letters, use |\textsc{...}|.                      \\
    \textsc{abcdefghijklmnopqrstuvwxyzABCDEFGHIJKLMNOPQRSTUVWXYZ0123456789}     \\
  \end{tabular}
\end{center}
Some of those text mode commands may seem redundant to you, and there are some you 
will never need to use, so don't worry if they look confusing. Remember that 
|\textnormal{...}| is the default and all you have to do to get it is start
typing; no command is necessary unless you have changed something (and that's usually 
difficult to do). These text font commands can be used together if you remember to 
nest the braces correctly. Look at the following table.

\begin{center}
  \begin{tabular}{l}
    Combined Text Mode Commands                                                      \\
    To get boldface san serif letters, use |\textbf{\textsf{...}}|.             \\
    \textbf{\textsf{abcdefghijklmnopqrstuvwxyzABCDEFGHIJKLMNOPQRSTUVWXYZ0123456789}} \\
    To get boldface italic letters, use |\textbf{\textit{...}}|.                \\
    \textbf{\textit{abcdefghijklmnopqrstuvwxyzABCDEFGHIJKLMNOPQRSTUVWXYZ0123456789}} \\
    To get boldface slanted letters, use |\textbf{\textsl{...}}|.               \\
    \textbf{\textsl{abcdefghijklmnopqrstuvwxyzABCDEFGHIJKLMNOPQRSTUVWXYZ0123456789}} \\
    %To get boldface small capital letters, use |\textbf{\textsc{...}}|.         \\
    %\textbf{\textsc{abcdefghijklmnopqrstuvwxyzABCDEFGHIJKLMNOPQRSTUVWXYZ0123456789}} \\
  \end{tabular}
\end{center}

\subsubsection{Emphasizing Words in Text Mode}
For semantic reasons, it's important to use |\emph{...}| when you want to 
emphasize a word. Use |\emph{vector}| to emphasize the word \emph{vector} or 
|\emph{momentum}| to emphasize the word \emph{momentum}. In most documents, 
|\emph{...}| defaults to italics, but it may default to something else when 
using a different document class. Let the document class do its work and you will 
have fewer things to remember. 

\subsubsection{Text Modal Font Commands}
Next, let's look as the various text modal commands you can use. I like to think
of them as \emph{switches} rather than \emph{commands} because they take effect 
immediately and stay in effect until you explicitly turn them off by activating 
another one, much like a light switch. As so, they do not take braced arguments, 
or indeed, any arguments at all. Look at the following table.

\begin{center}
  \begin{tabular}{l}
    Text Modal Commands                                                        \\
    Use |\normalfont| to switch to the default normal font.               \\
    \normalfont abcdefghijklmnopqrstuvwxyzABCDEFGHIJKLMNOPQRSTUVWXYZ0123456789
    \normalfont                                                                \\
    Use |\rmfamily| to switch to roman letters.                           \\
    \rmfamily abcdefghijklmnopqrstuvwxyzABCDEFGHIJKLMNOPQRSTUVWXYZ0123456789
    \normalfont                                                                \\
    Use |\sffamily| to switch to sans serif latters.                      \\
    \sffamily abcdefghijklmnopqrstuvwxyzABCDEFGHIJKLMNOPQRSTUVWXYZ0123456789
    \normalfont                                                                \\
    Use |\ttfamily| to switch to typewriter letters.                      \\
    \ttfamily abcdefghijklmnopqrstuvwxyzABCDEFGHIJKLMNOPQRSTUVWXYZ0123456789
    \normalfont                                                                \\
    Use |\mdseries| to switch to medium letters.                          \\
    \mdseries abcdefghijklmnopqrstuvwxyzABCDEFGHIJKLMNOPQRSTUVWXYZ0123456789
    \normalfont                                                                \\
    Use |\bfseries| to switch to boldface letters.                        \\
    \bfseries abcdefghijklmnopqrstuvwxyzABCDEFGHIJKLMNOPQRSTUVWXYZ0123456789
    \normalfont                                                                \\
    Use |\upshape| to switch to upright letters.                          \\
    \upshape abcdefghijklmnopqrstuvwxyzABCDEFGHIJKLMNOPQRSTUVWXYZ0123456789
    \normalfont                                                                \\
    Use |\itshape| to switch to italic letters.                           \\
    \itshape abcdefghijklmnopqrstuvwxyzABCDEFGHIJKLMNOPQRSTUVWXYZ0123456789
    \normalfont                                                                \\
    Use |\slshape| to switch to slanted letters.                          \\
    \slshape abcdefghijklmnopqrstuvwxyzABCDEFGHIJKLMNOPQRSTUVWXYZ0123456789
    \normalfont                                                                \\
    Use |\scshape| to switch to small capital letters.                    \\
    \scshape abcdefghijklmnopqrstuvwxyzABCDEFGHIJKLMNOPQRSTUVWXYZ0123456789
    \normalfont                                                                \\
  \end{tabular}
\end{center}

To illustrate how these switches work, the following unindented paragraph is in the 
default text font (it's just filler text). 

\noindent Lorem ipsum dolor sit amet, consectetur adipiscing elit. Duis egestas urna 
et dolor posuere, accumsan faucibus justo viverra. Pellentesque libero neque, maximus 
vitae placerat eu, luctus sit amet sapien. Class aptent taciti sociosqu ad litora 
torquent per conubia nostra, per inceptos himenaeos. Aenean vel nisl massa. Sed id 
velit tellus. Vivamus eu elit a erat aliquam auctor nec sed mi. Orci varius natoque 
penatibus et magnis dis parturient montes, nascetur ridiculus mus. Cras ut consequat 
purus. Fusce imperdiet scelerisque sagittis. Etiam maximus sagittis sapien. Proin mi 
metus, cursus a ex eget, maximus laoreet sapien. In hac habitasse platea dictumst. 
Morbi eget est dui. Cras posuere nisl quis leo facilisis, vel viverra augue sagittis. 
Nullam posuere, ex id efficitur mollis, velit nisi tincidunt augue, vel tempor orci 
nisi a justo.

The following unindented paragraph is the same filler text, but now preceded by the 
|\sffamily| switch.

\sffamily
\noindent Lorem ipsum dolor sit amet, consectetur adipiscing elit. Duis egestas urna 
et dolor posuere, accumsan faucibus justo viverra. Pellentesque libero neque, maximus 
vitae placerat eu, luctus sit amet sapien. Class aptent taciti sociosqu ad litora 
torquent per conubia nostra, per inceptos himenaeos. Aenean vel nisl massa. Sed id 
velit tellus. Vivamus eu elit a erat aliquam auctor nec sed mi. Orci varius natoque 
penatibus et magnis dis parturient montes, nascetur ridiculus mus. Cras ut consequat 
purus. Fusce imperdiet scelerisque sagittis. Etiam maximus sagittis sapien. Proin mi 
metus, cursus a ex eget, maximus laoreet sapien. In hac habitasse platea dictumst. 
Morbi eget est dui. Cras posuere nisl quis leo facilisis, vel viverra augue sagittis. 
Nullam posuere, ex id efficitur mollis, velit nisi tincidunt augue, vel tempor orci 
nisi a justo. 

Note that this (the current) paragraph is still typeset in sans serif letters because 
the |\sffamily| switch is still activated. We can deactivate it by activating 
another switch, like |\normalfont| \normalfont to get back to normal default 
text, or like |\bfseries| \bfseries to get bold letters, or even like 
|\itshape| \itshape to get italic letters. Note that we got both bold and italic 
letters in that last phrase, and in this sentence, because the |\bfseries| switch 
was still activated when we activated the |\itshape| switch. Activated switches 
can accumulate to give us new combinations of letter styles and shapes. We can return 
to the normal default font by activating the |\normalfont| switch \normalfont 
just like this. Now everything is back to normal.

\subsection{Math Mode Fonts}
Now we need to discuss math mode fonts, which are considerably more complicated than 
text mode fonts. There are math mode font \emph{commands}, but there are no math mode 
font \emph{switches}. Math mode fonts sometimes include lowercase and/or uppercase 
Greek letters, which are frequently used in mathematical notation. Math mode fonts 
go into effect whenever you enter in-line math mode with |\(...\)| or display 
math mode with |\[...\]|. Many \LaTeX\ tutorials use |$...$| for in-line 
math mode and |$$...$$| for display math mode. For various technical reasons, 
these are not recommended so please do not use them. Please consistently use 
|\(...\)| for in-line math mode and |\[...\]| for display math mode.

\subsubsection{Unicode Math Mode Font Commands}
In math mode, we use the 
\href{https://www.ctan.org/pkg/unicode-math}{\pkg{unicode-math}} 
package to create new font combinations 
that allow for greater flexibility than otherwise available. There is a tradeoff, and 
that is documents must be processed with the \lualatex\ engine. The math mode 
Unicode font commands available to you in this habitat are given in the following 
table. Note that all of these commands begin with |\sym<...>|. You may assume that 
characters not shown are not supported.
\begin{center}
  \begin{tabular}{l}
    |\(...\)| or |\[...\]| or |\symnormal{...}| give the default math mode font. 
    Use this for vector index notation. \\
    \( abcdefghijklmnopqrstuvwxyzABCDEFGHIJKLMNOPQRSTUVWXYZ0123456789 \) \\
    \( 
      \alpha\beta\gamma\delta\epsilon\varepsilon\zeta\eta\theta\vartheta\iota\kappa
      \lambda\mu\nu\xi\omicron\pi\varpi\rho\varrho\sigma\varsigma\tau\upsilon\phi\varphi
      \chi\psi\omega\Delta\Gamma\Theta\Lambda\Xi\Pi\Sigma\Upsilon\Phi\Psi\Omega 
    \) \\
    |\symbf{...}| gives boldface italic. Use this for coordinate-free 
    vectors and matrices. \\
    \( \symbf{abcdefghijklmnopqrstuvwxyzABCDEFGHIJKLMNOPQRSTUVWXYZ} \) \\
    \( \symbf{ 
       \alpha\beta\gamma\delta\epsilon\varepsilon\zeta\eta\theta\vartheta\iota\kappa
       \lambda\mu\nu\xi\omicron\pi\varpi\rho\varrho\sigma\varsigma\tau\upsilon\phi\varphi
       \chi\psi\omega\Delta\Gamma\Theta\Lambda\Xi\Pi\Sigma\Upsilon\Phi\Psi\Omega}
    \) \\
    |\symup{...}| gives upright (roman) serif. Use this to name particles.
    |\symup{\Omega}| (\(\symup{\Omega}\)) represents the ohm. \\
    \( \symup{abcdefghijklmnopqrstuvwxyzABCDEFGHIJKLMNOPQRSTUVWXYZ0123456789} \) \\
    \( \symup{ 
      \alpha\beta\gamma\delta\epsilon\varepsilon\zeta\eta\theta\vartheta\iota\kappa
      \lambda\mu\nu\xi\omicron\pi\varpi\rho\varrho\sigma\varsigma\tau\upsilon\phi\varphi
      \chi\psi\omega\Delta\Gamma\Theta\Lambda\Xi\Pi\Sigma\Upsilon\Phi\Psi\Omega}
    \) \\
    |\symbfup{...}| gives boldface upright serif. \\
    \( \symbfup{abcdefghijklmnopqrstuvwxyzABCDEFGHIJKLMNOPQRSTUVWXYZ0123456789} \) \\
    \( \symbfup{ 
       \alpha\beta\gamma\delta\epsilon\varepsilon\zeta\eta\theta\vartheta\iota\kappa
       \lambda\mu\nu\xi\omicron\pi\varpi\rho\varrho\sigma\varsigma\tau\upsilon\phi\varphi
       \chi\psi\omega\Delta\Gamma\Theta\Lambda\Xi\Pi\Sigma\Upsilon\Phi\Psi\Omega}
    \) \\
    |\symsfup{...}| gives sans serif upright. Use this for physical dimensions. \\
    \( \symsfup{abcdefghijklmnopqrstuvwxyzABCDEFGHIJKLMNOPQRSTUVWXYZ0123456789} \) \\
    |\symbfsfup{...}| gives boldface sans serif upright. \\
    \( \symbfsfup{abcdefghijklmnopqrstuvwxyzABCDEFGHIJKLMNOPQRSTUVWXYZ0123456789} \) \\
    \( \symbfsfup{ 
       \alpha\beta\gamma\delta\epsilon\varepsilon\zeta\eta\theta\vartheta\iota\kappa
       \lambda\mu\nu\xi\omicron\pi\varpi\rho\varrho\sigma\varsigma\tau\upsilon\phi\varphi
       \chi\psi\omega\Delta\Gamma\Theta\Lambda\Xi\Pi\Sigma\Upsilon\Phi\Psi\Omega}
    \) \\
    |\symsfit{...}| gives sans serif italic. Use this for tensor index notation. \\
    \( \symsfit{abcdefghijklmnopqrstuvwxyzABCDEFGHIJKLMNOPQRSTUVWXYZ} \) \\
    |\symbfsfit{...}| gives boldface sans serif italic. Use this for 
    coordinate-free tensors.  \\
    \( \symbfsfit{abcdefghijklmnopqrstuvwxyzABCDEFGHIJKLMNOPQRSTUVWXYZ} \) \\
    |\symcal{...}| and|\symbfcal{...}| give calligraphic. Use these for 
    points on a manifold or \\
    naming coordinate systems. \\
    \( \symcal{ABCDEFGHIJKLMNOPQRSTUVWXYZ} \) \\
    \( \symbfcal{ABCDEFGHIJKLMNOPQRSTUVWXYZ} \) \\
    |\symscr{...}| and |\symbfscr{...}| give script. Use these for
    naming spacetime events. \\
    \( \symscr{abcdefghijklmnopqrstuvwxyzABCDEFGHIJKLMNOPQRSTUVWXYZ} \) \\
    \( \symbfscr{abcdefghijklmnopqrstuvwxyzABCDEFGHIJKLMNOPQRSTUVWXYZ} \) \\
    |\symtt{...}| gives typewriter. \\
    \( \symtt{abcdefghijklmnopqrstuvwxyzABCDEFGHIJKLMNOPQRSTUVWXYZ0123456789} \) \\
    |\symfrak{...}| and |\symbffrak{...}| give Fraktur. \\
    \( \symfrak{abcdefghijklmnopqrstuvwxyzABCDEFGHIJKLMNOPQRSTUVWXYZ} \) \\
    \( \symbffrak{abcdefghijklmnopqrstuvwxyzABCDEFGHIJKLMNOPQRSTUVWXYZ} \) \\
    |\symbb{...}| gives blackboard. \\
    \( \symbb{abcdefghijklmnopqrstuvwxyzABCDEFGHIJKLMNOPQRSTUVWXYZ0123456789} \) \\
    |\symbbit{...}| gives blackboard italic. I have no idea why it's defined for only 
    five letters.\\
    \( \symbbit{deijD} \) \\
  \end{tabular}
\end{center}
The zero vector is |\symbfup{0}| or \(\symbfup{0}\). It is aliased as |\zerovector|, 
which also gives \(\zerovector\). As you can see, many more characters are available 
using the Unicode commands. Also, notice the semantic naming of each typeface and style.

\subsubsection{A Few More Mathematical Symbols}
There's a handful of other mathematical symbols you will need, and here they are. Don't worry 
if you've never seen any or all of these before. The third row of symbols is unique to \pkg{mandi}.
\begin{center}
  \begin{tabular}{l}
    \(\nabla\,\mdlgwhtsquare\,\partial\,\otimes\,\times\,\cdot\,\bullet\) \\
    \(\symbf{\nabla}\,\symbf{\partial}\) \\
    \(\nabla\indices{_{\veccomp{u}}}\,\nabla\indices{_{\tencomp{u}}}\) \\
    \(\slot\,\slot*[\veccomp{a}]\,\contraction{(1,2)}\,\contraction{(1,3)(2,4)}\,
      \contraction*{(1,2)}\,\contraction*{(1,3)(2,4)}\) \\
  \end{tabular}
\end{center}

\subsubsection{Unicode Input}
With Unicode support, you can use characters direction from your keyboard. The following
example of Maxwell's equations uses keyboard characters typed directly. 
\begin{gather*}
  ∇∙𝑬=\frac{𝜌}{𝜀_o} \\
    ∇∙𝑩=0 \\
    ∇×𝑬=−\frac{∂𝑩}{∂𝑡} \\
    ∇×𝑩=𝜇_o \biggl(I + 𝜀_o \frac{∂𝑬}{∂𝑡}\biggr)
\end{gather*}

There is an intelligent exterior derivative operator. 
\[
  \dd*\omega
\]
% Back to wider margins.
\restoregeometry

\section{The Unit Engine}
The seven fundamental SI quantities are displacement (\displacement{3}), time (\duration{3}),
electric current (\current{1.2}), thermodynamic temperature (\temperature{3}), 
amount (\amount{4}), and luminous intensity (\luminous{2}). Momentum (\momentum{4}) is a very 
important quantity.

\hereusealternativeunit{\momentum{4}}

\section{Column and Row Vectors}
\subsection{Column Vectors}
\[
  \colvec{0,1,2,3}\colvec{\tencomp*{a}_1,\tencomp*{a}_2,\tencomp*{a}_2}
\]
\subsection{Row Vectors}
\[
  \rowvec{0,1,2,3}\rowvec{\tencomp*{a}_1,\tencomp*{a}_2,\tencomp*{a}_2}
\]

\section{Symbolic Vectors}
\subsection{Notation Conventions}
Typesetting symbolic vectors is ubiquitous in in physics, yet most sources are not consistent in
notation compared to other sources. Using arrows to denote vector quantities, as with |\vec*{p}|
(note the |*|), which gives \vec*{p}, seems the standard in introductory physics courses, but 
this isn't seen in higher level resources, and certainly not in published articles. The standard, 
which doesn't seem to be observed, is that vectors are typeset in boldface italic, as with 
|\vec{p}|, which gives \vec{p}. Physics teachers know students struggle with using arrows to
denote vector quantities, and most students never correctly learn it. After all, why should they learn 
something that isn't used consistently, or even at all? There is really just one practical reason 
for using arrows rather than boldface, and that is the quaint notion of writing on a whiteboard (or a 
chalkboard if anyone remembers those). We really should move with the times and enforce the professional 
convention for our students. So in \pkg{mandi} we do exactly this, but allow flexibility in notation for 
those who insist upon having it. By default, the standard \LaTeX\ |\vec{...}| command is redefined
to, by default, typeset symbols for vector quantities in boldface italic. The starred form of the
command, |\vec*{...}|, typesets them with arrows. Note that getting arrow notation is a bit
more difficult (because you must type the |*|) than the default boldface italic notation, and
this is a subtle psychological tactic on my part; the new notation takes less effort.

\subsection{Symbolic Notation}
The command |\dirvect{...}| typesets the symbol for a vector's direction.
|\dirvect{r}| gives \( \dirvect{r} \) and as you might expect by now |\dirvect*{r}|
gives \( \dirvect*{r} \). A vector's magnitude is typeset with |\norm{...}|. As you may
predict, |\norm{\vec{r}}| gives \( \norm{\vec{r}} \) and |\norm{\vec*{r}}| gives 
\( \norm{\vec*{r}} \). Any vector can be \emph{factored} into its magnitude and direction. So
\( \vec{p} = \norm{\vec{p}}\dirvect{p} \) or \( \vec*{p} = \norm{\vec*{p}}\dirvect*{p} \).

The command |\magvect{...}| typesets the symbol for a vector's magnitude.
|\magvect{r}| gives \( \magvect{r} \) and of course |\magvect*{r}| gives
\( \magvect*{r} \).

Finally, the command |\Dvect{...}| typesets the symbol for the change in a vector. |\Dvect{r}|
gives \( \Dvect{r} \). For arrow notation, use the starred form. |\Dvect*{r}| gives
\( \Dvect*{r} \). 

\subsection{Embellishments}
Again, I state that notation should help the reader and for physics students, unambiguously help 
guide one's thinking. This frequently requires use of embellishments, or additions to basic notation.
For example, consider a rock moving through space. One can speak of the rock's momentum, and 
notate it as |\vec{p}_{rock}|, which gives 
\[
\vec{p}_{rock}
\]
as predicted. But wait, the subscript doesn't look right. It is typeset in italic when it shouldn't 
be. So instead, we write |\vec{p}_{\mathup{rock}}|, which gives 
\[
\vec{p}_{\mathup{rock}}
\]
which looks much better. Why does this make a difference? The answer is that the |\vec{...}| 
command internally takes place in \emph{math mode}, which means all characters are, by default, typeset 
in italic. Words are not intended to be written in math mode. So if we want to typeset an actual piece 
of text, you must mark it up as such using a |\mathup{...}| command.

So now you can correctly typeset an initial momentum and final momentum symbolically as \newline
|\vec{p}_{\mathup{initial}}| and |\vec{p}_{\mathup{final}}|, which
give
\[
\vec{p}_{\mathup{initial}} \text{ and } \vec{p}_{\mathup{final}}
\]
exactly as expected. If you want something other than actual text as a subscript, just supply it 
without additional markup. So you could write, for example, |\vec{p}_{1}| and get
\[
\vec{p}_{1}
\]
as expected. The braces around the subscript are strictly not required if the subscript is just one
character, but it is a good habit to write them all the time, so I do.

You can add superscripts just as easily as you can add subscripts, using a caret (|^|) to 
denote the superscript. Writing |\vec{p}^{\mathup{rock}}| gives
\[
\vec{p}^{\mathup{rock}}
\]
just like that. All the comments about subscripts above apply to superscripts. The best part is
that you can effortlessly mix subscripts and superscripts. Writing \newline
|\vec{p}_{\mathup{final}}^{\mathup{rock}}| or \newline
|\vec{p}^{\mathup{rock}}_{\mathup{final}}| gives
\[
\vec{p}_{\mathup{final}}^{\mathup{rock}} \text{ and } \vec{p}^{\mathup{rock}}_{\mathup{final}}
\]
and note that the order of the flourishes doesn't matter. The underlying \LaTeX3\ programming 
layer makes this almost trivial to implement in \pkg{mandi} now. Using text to embellish a symbolic
vector takes a bit more effort, which is a subtle warning to use it sparingly. Simply changing
|\vec{...}| to |\vec*{...}| gives the same results with arrow notation and gives
\[
\vec*{p}_{\mathup{final}}^{\mathup{rock}} \text{ and } \vec*{p}^{\mathup{rock}}_{\mathup{final}}
\]
as expected.

\begin{center}
\bfseries It is very important to always that when in math mode, always, without exception, wrap words 
used in subscripts and superscripts in |\mathup{...}|. This does not apply to mathematical symbols,
only to actual words or any text that is not mathematical notation.\normalfont
\end{center}

Of course you can mix mathematical notation and text in the flourishes. As an example, one could
represent an alpha particle's final momentum by writing \newline
|\vec{p}_{\symup{\alpha},\mathup{final}}| which typesets as
\[
\vec{p}_{\symup{\alpha},\mathup{final}}
\]
as beautifully as expected. If you want arrow notation, you would write \newline
|\vec*{p}_{\symup{\alpha},\mathup{final}}| which typesets as
\[
\vec*{p}_{\symup{\alpha},\mathup{final}}
\]
as expected. For many reasons, I think the boldface notation looks better. Finally, note that
|\symup{...}| is used here since upright Greek letters are used to name subatomic particles.

\[
  \norm*{\vec{p}_{\mathup{final}}} \qquad \pens*{\frac{1}{4\pi\varepsilon_o}}
\]
\begin{align}
  \Delta\vec{r}  &= \vec{r}_{\mathup{final}} - \vec{r}_{\mathup{initial}} \reason{definition} \\
  \Delta\vec*{r} &= \vec*{r}_{\mathup{final}} - \vec*{r}_{\mathup{initial}} 
    \reason{\lipsum[2][2-4]} \\
  \Delta\vec{r}  &= \vec{r}_{\mathup{final}} - \vec{r}_{\mathup{initial}} \reason{definition} \\
  \Delta\vec{r}  &= \vec{r}_{\mathup{final}} - \vec{r}_{\mathup{initial}} \reason{definition} \\
  \norm{\vec{r}} &= \sqrt{\vec{r}\cdot\vec{r}}
\end{align}

\begin{gather*}
\inpens[\frac{1}{4\pi\epsilon_o}] \qquad \inpens \\
\indims[\frac{1}{4\pi\epsilon_o}] \qquad \indims \\
\inabsv[\frac{1}{4\pi\epsilon_o}] \qquad \inabsv \\
\innorm[\frac{1}{4\pi\epsilon_o}] \qquad \innorm \\
\pens*{\frac{1}{4\pi\epsilon_o}}  \qquad \pens{} \\
\dims*{\frac{1}{4\pi\epsilon_o}}  \qquad \dims{} \\
\absv*{\frac{1}{4\pi\epsilon_o}}  \qquad \absv{} \\
\norm*{\frac{1}{4\pi\epsilon_o}}  \qquad \norm{} \\
\end{gather*}

\subsection{Coordinate-Free Notation}
The arrow notation used in introductory physics is a compromise between a notation that ostensibly
helps students remember the quantity in question had an associate direction and a notation that
is easily reproducible by hand on a whiteboard. As hard as we try, students rarely use arrow
notation for vectors correctly, if at all. You don't find arrow notation for vectors much at 
all in the literature outside of introductory textbooks. Most sources use boldface, and while 
that is harder to use on a whiteboard, it's trivial to do in \LaTeX\ so why not adopt it as
the expectation?

Use the |\veccomp{...}| command to get coordinate-free notation for vectors. |\veccomp{p}| 
gives \veccomp{p}. This notation isn't intended to be used with embellishments, but it can be done 
as long as you are in math mode.

\subsection{Index Notation}
Use the |\veccomp*{...}| command to get index notation for vectors. |\veccomp*{p}|
gives \veccomp*{p}. Where is the index? You have to supply it using the |\indices{...}|
command from the 
\href{https://www.ctan.org/pkg/tensor}{\pkg{tensor}} 
package, which \pkg{mandi} loads for you. So to get the symbol for momentum in index notation you would 
use |\veccomp*{p}\indices{^i}| which gives \( \veccomp*{p}\indices{^i} \). Now, you may realize that 
|\veccomp*{p}^i| gives the same result, and it does indeed give \( \veccomp*{p}^i \) as expected, so 
why bother with the |\indices{...}| command? The answer is that the |\indices{...}| command plays an 
important role in tensor index notation, which includes vector index notation, and you need to be 
comfortable  using it. Use |\veccomp{e}| to get a basis vector, which gives \veccomp{e} as expected. 
Basis vectors must be labeled, and I recommend using the |indices{...}| command, but both
|\veccomp{e}\indices{_i}| and |\veccomp{e}_i| give the same result, 
\( \veccomp{e}\indices{_i} \) as expected. So to get the full symbol for a vector in a particular
basis you would use |\veccomp*{p}_i\veccomp{e}^i| or 
|\veccomp*{p}\indices{^i}\veccomp{e}\indices{_i}|, both of which give 
\( \veccomp*{p}\indices{^i}\veccomp{e}\indices{_i} \).

\section{Highlighting Mathematics}
A new feature to \pkg{mandi} is a command for highlighting mathematics.
\begin{align*}
  (\Delta s)^2 &= -(\Delta t)^2 + (\Delta x)^2 + (\Delta y)^2 + (\Delta z)^2 \\
  (\Delta s)^2 &= \hilite{-(\Delta t)^2 + (\Delta x)^2}[rounded rectangle] + (\Delta y)^2 + (\Delta z)^2 \\
  (\Delta s)^2 &= \hilite{-(\Delta t)^2 + (\Delta x)^2}[rectangle] + (\Delta y)^2 + (\Delta z)^2 \\
  (\Delta s)^2 &= \hilite{-(\Delta t)^2 + (\Delta x)^2}[ellipse] + (\Delta y)^2 + (\Delta z)^2 \\
  (\Delta s)^{\hilite{2}[circle]} &= \hilite[green]{-}[circle](\Delta t)^{\hilite[cyan]{2}[circle]}+
    (\Delta x)^{\hilite[orange]{2}[circle]} + (\Delta y)^{\hilite[blue!50]{2}[circle]} +
    (\Delta z)^{\hilite[violet!45]{2}[circle]}
\end{align*}
Here is another example.
\begin{align*}
  \Delta\mathbf{p} &= \mathbf{F}_{\mathup{net,sys}}\,\Delta t \\
  \hilite[orange]{\Delta\mathbf{p}}[circle] &= \mathbf{F}_{\mathup{net,sys}}\,\Delta t \\
  \Delta\mathbf{p} &= \hilite[yellow!50]{\mathbf{F}_{\mathup{net,sys}}}[rounded rectangle]\,\Delta t \\
  \Delta\mathbf{p} &= \mathbf{F}_{\mathup{net,sys}}\,\hilite[olive!50]{\Delta t}[rectangle] \\
  \Delta\mathbf{p} &= \hilite[cyan!50]{\mathbf{F}_{\mathup{net,sys}}\,\Delta t}[ellipse] \\
  \hilite{\Delta\mathbf{p}}[rectangle] &= \mathbf{F}_{\mathup{net,sys}}\,\Delta t
\end{align*}

Highlighting also works in in-line math mode like \( a^2 + \hilite{b^2}[circle] = c^2 \).

\section{Tensor Notation}
A tensor has a valence \valence{m}{n} by definition. A vector is a \valence{1}{0} tensor and a
one-form is a \valence{0}{1} tensor.

\section{\GlowScript\ and \VPython\ Program Listings}
\subsection{The \texttt{glowscriptblock} Environment}
\GlowScript\ program listings are improved and look better. Clicking anywhere on the actual 
code takes the reader to the relevant source code on \href{https://www.glowscript.org}{\GlowScriptorg}
for read-only access using the URL provided by the user. To minimize the chances of a program being 
broken over a pagebreak, each listing begins on a new page. This is a big change from the previous 
version of \pkg{mandi} and may be annoying for small programs, but I think the clarity of beginning a
new listing on a new page is important.

\begin{glowscriptblock}[https://google.com]{A \GlowScript\ program.}{pref1}
GlowScript 3.0 VPython

scene.width = 400
scene.height = 760
# constants and data
g = 9.8       # m/s^2
mball = 0.03  # kg
Lo = 0.26     # m
ks = 1.8      # N/m
deltat = 0.01 #s 

# objects (origin is at ceiling)
ceiling = box(pos=vector(0,0,0), length=0.2, height=0.01, width=0.2)
ball = sphere(pos=vector(0,-0.3,0),radius=0.025,color=color.orange)
spring = helix(pos=ceiling.pos, axis=ball.pos-ceiling.pos,
         color=color.cyan,thickness=0.003,coils=40,radius=0.010)

# initial values
pball = mball * vector(0,0,0)      # kg m/s
Fgrav = mball * g * vector(0,-1,0) # N
t = 0

# improve the display
scene.autoscale = False        # turn off automatic camera zoom
scene.center = vector(0,-Lo,0) # move camera down
scene.waitfor('click')         # wait for a mouse click

# initial calculation loop
# calculation loop
#while t < 10:
#    rate(100)
#    Fnet = Fgrav
#    pball = pball + Fnet * deltat
#    ball.pos = ball.pos + (pball / mball) * deltat
#    spring.axis = ball.pos - ceiling.pos
#    t = t + deltat
    
# modified initial calculation loop
# calculation loop
while t < 10:
    rate(100)
    # we need the stretch
    s = mag(ball.pos) - Lo
    # we need the spring force
    Fspring = ks * s * -norm(spring.axis)
    Fnet = Fgrav + Fspring
    pball = pball + Fnet * deltat
    ball.pos = ball.pos + (pball / mball) * deltat
    spring.axis = ball.pos - ceiling.pos
    t = t + deltat
\end{glowscriptblock}

nameref-link to listing: \nameref{pref1}

ref-link to listing: \ref{pref1}

pageref-link to listing: \pageref{pref1}

autoref-link to listing: \autoref{pref1}

It may be annoying for a small program listing like this, but again, I think the clarity is 
important, especially for beginners. 

\begin{glowscriptblock}[https://google.com]{79 Columns}{pref100}
1234567890123456789012345678901234567890123456789012345678901234567890123456789
1234567890123456789012345678901234567890123456789012345678901234567890123456789
\end{glowscriptblock}

nameref-link to listing: \nameref{pref100}

ref-link to listing: \ref{pref100}

pageref-link to listing: \pageref{pref100}

autoref-link to listing: \autoref{pref100}

\subsection{The \texttt{glowscriptblock} Environment With \VPython\ Programs}
\href{https://vpython.org}{\VPython} programs are also nicely formatted but lack the hyperlink capability
because they live on the user's local computer and not online. Since the |glowscriptblock| environment 
turns the code into a hyperlink, it makes no sense semantically to use this environment for \VPython\ 
programs. Still, it is sometimes helpful to show a block of code this way without turning the code into 
a hyperlink. Use |glowscriptblock| without the optional argument to typeset a block of \VPython\ code. 
|\begin{glowscriptblock}{A \VPython\ program}{pref2}| gives

\begin{glowscriptblock}{A \VPython\ program}{pref2}
from vpython import *

scene.width = 400
scene.height = 760
# constants and data
g = 9.8       # m/s^2
mball = 0.03  # kg
Lo = 0.26     # m
ks = 1.8      # N/m
deltat = 0.01 #s 

# objects (origin is at ceiling)
ceiling = box(pos=vector(0,0,0), length=0.2, height=0.01, width=0.2)
ball = sphere(pos=vector(0,-0.3,0),radius=0.025,color=color.orange)
spring = helix(pos=ceiling.pos, axis=ball.pos-ceiling.pos,
         color=color.cyan,thickness=0.003,coils=40,radius=0.010)

# initial values
pball = mball * vector(0,0,0)      # kg m/s
Fgrav = mball * g * vector(0,-1,0) # N
t = 0

# improve the display
scene.autoscale = False        # turn off automatic camera zoom
scene.center = vector(0,-Lo,0) # move camera down
scene.waitfor('click')         # wait for a mouse click

# initial calculation loop
# calculation loop
#while t < 10:
#    rate(100)
#    Fnet = Fgrav
#    pball = pball + Fnet * deltat
#    ball.pos = ball.pos + (pball / mball) * deltat
#    spring.axis = ball.pos - ceiling.pos
#    t = t + deltat
    
# modified initial calculation loop
# calculation loop
while t < 10:
    rate(100)
    # we need the stretch
    s = mag(ball.pos) - Lo
    # we need the spring force
    Fspring = ks * s * -norm(spring.axis)
    Fnet = Fgrav + Fspring
    pball = pball + Fnet * deltat
    ball.pos = ball.pos + (pball / mball) * deltat
    spring.axis = ball.pos - ceiling.pos
    t = t + deltat
\end{glowscriptblock}

nameref-link to listing: \nameref{pref2}

ref-link to listing: \ref{pref2}

pageref-link to listing: \pageref{pref2}

autoref-link to listing: \autoref{pref2}

\subsection{In-line \GlowScript\ and \VPython\ Code}
In-line code can be included as before using |\glowscriptline{...}| or 
|\vpythonline{...}|, depending on which is semantically appropriate. Using \newline
|\glowscriptline{Earth = sphere(color=color.blue)}| gives \newline
\glowscriptline{Earth = sphere(pos=vector(1,1,1),radius=2,color=color.blue)} and \newline
|\vpythonline{Sun = sphere(pos=vector(1,1,1),radius=2,color=color.yellow)}| gives \newline
\vpythonline{Sun = sphere(pos=vector(1,1,1),radius=2,color=color.yellow)}.

\subsection{Including a \VPython\ File}
A \VPython\ file can be included using |\vpythonfile{...}{...}{...}|. The file 
should be in the same folder as the document being typeset, and a caption and a reference 
label must be provided. Again, the typeset listing begins on a new page.

\vpythonfile{Sample Program}{vlabel1}{MyProgram.py}

nameref-link to listing: \nameref{vlabel1}

ref-link to listing: \ref{vlabel1}

pageref-link to listing: \pageref{vlabel1}

autoref-link to listing: \autoref{vlabel1}

\section{Commands Specific to \emph{Matter \& Interactions}}
These commands are defined in an optional accessory package called \pkg{mandiexp}. That package's name 
is subject to change.
\begin{align*}
 &\momentumprinciple         \reason{momentum principle}                      \\
 &\momentumprinciple*        \reason{momentum principle, update form}         \\
 &\energyprinciple           \reason{energy principle}                        \\
 &\energyprinciple*          \reason{energy principle, update form}           \\
 &\angularmomentumprinciple  \reason{angular momentum principle}              \\
 &\angularmomentumprinciple* \reason{angular momentum principle, update form}               
\end{align*}
\begin{gather*}
  \systemenergy   \quad\systemenergy[\symup{final}]   \quad\systemenergy[\symup{initial}]
    \quad\changein\systemenergy \\
  \particleenergy \quad\particleenergy[\symup{final}] \quad\particleenergy[\symup{initial}] 
    \quad\changein\particleenergy \\
  \restenergy     \quad\restenergy[\symup{final}]     \quad\restenergy[\symup{initial}]     
    \quad\changein\restenergy \\
  \internalenergy \quad\internalenergy[\symup{final}] \quad\internalenergy[\symup{initial}] 
    \quad\changein\internalenergy \\
  \chemicalenergy \quad\chemicalenergy[\symup{final}] \quad\chemicalenergy[\symup{initial}] 
    \quad\changein\chemicalenergy \\
  \thermalenergy  \quad\thermalenergy[\symup{final}]  \quad\thermalenergy[\symup{initial}]  
    \quad\changein\thermalenergy \\
  \photonenergy   \quad\photonenergy[\symup{final}]   \quad\photonenergy[\symup{initial}]   
    \quad\changein\photonenergy \\
  \energyof{\symup{electron}}
    \quad\energyof{\symup{electron}}[\symup{final}]
    \quad\energyof{\symup{electron}}[\symup{initial}] 
    \quad\changein\energyof{\symup{electron}} \\
  \energyof{\symup{\beta}}
    \quad\energyof{\symup{\beta}}[\symup{final}]
    \quad\energyof{\symup{\beta}}[\symup{initial}] 
    \quad\changein\energyof{\symup{\beta}} \\
  \translationalkineticenergy  \quad\translationalkineticenergy[\symup{final}] 
    \quad\translationalkineticenergy[\symup{initial}] \quad\changein\translationalkineticenergy \\
  \translationalkineticenergy* \quad\translationalkineticenergy*[\symup{final}] 
    \quad\translationalkineticenergy*[\symup{initial}] \quad\changein\translationalkineticenergy* \\
  \rotationalkineticenergy     \quad\rotationalkineticenergy[\symup{final}] 
    \quad\rotationalkineticenergy[\symup{initial}] \quad\changein\rotationalkineticenergy \\
  \rotationalkineticenergy*    \quad\rotationalkineticenergy*[\symup{final}] 
    \quad\rotationalkineticenergy*[\symup{initial}] \quad\changein\rotationalkineticenergy* \\
  \vibrationalkineticenergy    \quad\vibrationalkineticenergy[\symup{final}] 
    \quad\vibrationalkineticenergy[\symup{initial}] \quad\changein\vibrationalkineticenergy \\
  \vibrationalkineticenergy*   \quad\vibrationalkineticenergy*[\symup{final}] 
    \quad\vibrationalkineticenergy*[\symup{initial}] \quad\changein\vibrationalkineticenergy* \\
  \gravitationalpotentialenergy \quad\gravitationalpotentialenergy[\symup{final}] 
    \quad\gravitationalpotentialenergy[\symup{initial}] \quad\changein\gravitationalpotentialenergy \\
  \electricpotentialenergy \quad\electricpotentialenergy[\symup{final}] 
    \quad\electricpotentialenergy[\symup{initial}] \quad\changein\electricpotentialenergy \\
  \springpotentialenergy \quad\springpotentialenergy[\symup{final}] 
    \quad\springpotentialenergy[\symup{initial}] \quad\changein\springpotentialenergy
\end{gather*}
These commands are intended to be semantically helpful.

\end{document}

